%!TEX TS-program = xelatex
%!TEX encoding = UTF-8 Unicode
% Awesome CV LaTeX Template for CV/Resume
%
% This template has been downloaded from:
% https://github.com/posquit0/Awesome-CV
%
% Author:
% Claud D. Park <posquit0.bj@gmail.com>
% http://www.posquit0.com
%
%
% Adapted to be an Rmarkdown template by Mitchell O'Hara-Wild
% 23 November 2018
%
% Template license:
% CC BY-SA 4.0 (https://creativecommons.org/licenses/by-sa/4.0/)
%
%-------------------------------------------------------------------------------
% CONFIGURATIONS
%-------------------------------------------------------------------------------
% A4 paper size by default, use 'letterpaper' for US letter
\documentclass[11pt,a4paper,]{awesome-cv}

% Configure page margins with geometry
\usepackage{geometry}
\geometry{left=1.4cm, top=.8cm, right=1.4cm, bottom=1.8cm, footskip=.5cm}


% Specify the location of the included fonts
\fontdir[fonts/]

% Color for highlights
% Awesome Colors: awesome-emerald, awesome-skyblue, awesome-red, awesome-pink, awesome-orange
%                 awesome-nephritis, awesome-concrete, awesome-darknight

\definecolor{awesome}{HTML}{009ACD}

% Colors for text
% Uncomment if you would like to specify your own color
% \definecolor{darktext}{HTML}{414141}
% \definecolor{text}{HTML}{333333}
% \definecolor{graytext}{HTML}{5D5D5D}
% \definecolor{lighttext}{HTML}{999999}

% Set false if you don't want to highlight section with awesome color
\setbool{acvSectionColorHighlight}{true}

% If you would like to change the social information separator from a pipe (|) to something else
\renewcommand{\acvHeaderSocialSep}{\quad\textbar\quad}

\def\endfirstpage{\newpage}

%-------------------------------------------------------------------------------
%	PERSONAL INFORMATION
%	Comment any of the lines below if they are not required
%-------------------------------------------------------------------------------
% Available options: circle|rectangle,edge/noedge,left/right

\name{Miriam}{Lerma}

\position{Postdoctoral researcher}
\address{Forschung und Technologiezentrum Westküste Der
Christian-Albrecht-Universität zu Kiel}

\email{\href{mailto:lerma@ftz-west.uni-kiel.de}{\nolinkurl{lerma@ftz-west.uni-kiel.de}}}
\homepage{www.miriam-lerma.com}

% \gitlab{gitlab-id}
% \stackoverflow{SO-id}{SO-name}
% \skype{skype-id}
% \reddit{reddit-id}


\usepackage{booktabs}

\providecommand{\tightlist}{%
	\setlength{\itemsep}{0pt}\setlength{\parskip}{0pt}}

%------------------------------------------------------------------------------


\definecolor{light-gray}{gray}{0.95}

\usepackage{fontawesome} \usepackage{colortbl}

\arrayrulecolor{white}

% Pandoc CSL macros
\newlength{\cslhangindent}
\setlength{\cslhangindent}{1.5em}
\newlength{\csllabelwidth}
\setlength{\csllabelwidth}{2em}
\newenvironment{CSLReferences}[3] % #1 hanging-ident, #2 entry spacing
 {% don't indent paragraphs
  \setlength{\parindent}{0pt}
  % turn on hanging indent if param 1 is 1
  \ifodd #1 \everypar{\setlength{\hangindent}{\cslhangindent}}\ignorespaces\fi
  % set entry spacing
  \ifnum #2 > 0
  \setlength{\parskip}{#2\baselineskip}
  \fi
 }%
 {}
\usepackage{calc}
\newcommand{\CSLBlock}[1]{#1\hfill\break}
\newcommand{\CSLLeftMargin}[1]{\parbox[t]{\csllabelwidth}{\honortitlestyle{#1}}}
\newcommand{\CSLRightInline}[1]{\parbox[t]{\linewidth - \csllabelwidth}{\honordatestyle{#1}}}
\newcommand{\CSLIndent}[1]{\hspace{\cslhangindent}#1}

\begin{document}

% Print the header with above personal informations
% Give optional argument to change alignment(C: center, L: left, R: right)
\makecvheader

% Print the footer with 3 arguments(<left>, <center>, <right>)
% Leave any of these blank if they are not needed
% 2019-02-14 Chris Umphlett - add flexibility to the document name in footer, rather than have it be static Curriculum Vitae


%-------------------------------------------------------------------------------
%	CV/RESUME CONTENT
%	Each section is imported separately, open each file in turn to modify content
%------------------------------------------------------------------------------



\hypertarget{research-interest}{%
\section{Research interest}\label{research-interest}}

\begin{itemize}
\tightlist
\item
  I am a postdoctoral researcher at West Coast Research and Technology
  Center FTZ in Büsum, Germany and part of the Seevogelmonitoring team
  at the Dachverband Deutscher Avifaunisten DDA. My research interest
  include ecology, conservation, physiology and ecotoxicology.
\end{itemize}

\hypertarget{education}{%
\section{Education}\label{education}}

\begin{cventries}
    \cventry{Christian-Albrecht-Universität zu Kiel}{Doctoral studies}{2015-20}{Kiel, Germany}{}\vspace{-4.0mm}
    \cventry{Universidad Nacional Autonoma de México}{Master in Science in Marine Biology }{2011-13}{Mazatlan, Mexico}{}\vspace{-4.0mm}
    \cventry{Universidad de Guadalajara }{Bachelor in Science in Biology }{2005-09}{Guadalajara, Mexico}{}\vspace{-4.0mm}
\end{cventries}

\hypertarget{work-experience}{%
\section{Work experience}\label{work-experience}}

\begin{cventries}
    \cventry{Dachverband Deutscher Avifaunisten}{Research associate}{2021}{ }{\begin{cvitems}
\item Project: MarBird
\end{cvitems}}
    \cventry{Forschung und Technologiezentrum Westküste}{Postdoc researcher}{2021}{ }{\begin{cvitems}
\item Project: TopMarine
\end{cvitems}}
    \cventry{Fondo Noroeste AC (Civil association)}{Field technician}{2015}{ }{\begin{cvitems}
\item Project: Composition, abundance and habitat use of birds on shrimp farms on Bahia de Santa Maria, Sinaloa
\end{cvitems}}
    \cventry{CONANP (National Commission of Natural Protected Areas)}{Field technician}{2014}{ }{\begin{cvitems}
\item Project: Biological monitoring on Natural Protected Areas (PROMOBI 2014) (focused on Shorebirds)
\end{cvitems}}
    \cventry{CONANP (National Commission of Natural Protected Areas)}{Field technician}{2014}{ }{\begin{cvitems}
\item Project: Conservation of Species on Risk (PROCER 2014) (focused on American Oystercatcher)
\end{cvitems}}
    \cventry{UNAM-FONATUR (National Autonomous University of Mexico- National Trust Fund for Tourism Development)}{Advisor}{2014}{ }{\begin{cvitems}
\item Subproject: Fauna monitoring on project CIP Pacific Coast on Escuinapa, Sinaloa
\end{cvitems}}
    \cventry{CONANP-IGC (National Commission of Natural Protected Areas-Gulf of California Islands)}{Instructor}{2014}{ }{\begin{cvitems}
\item Course:Introduction to studies on birds
\end{cvitems}}
    \cventry{Sonoran Joint Venture (Binational Bird Conservation)}{Field technician}{2012}{ }{\begin{cvitems}
\item Project: Monitoring seabirds in Santa Maria Bay, Sinaloa: Population size, nesting habitat, reproductive success and diet
\end{cvitems}}
    \cventry{CONACYT (National Council on Science and Technology)}{Field technician}{2012}{ }{\begin{cvitems}
\item Project: Seabirds as environmental indicators in coastal systems of economic importance
\end{cvitems}}
    \cventry{Grupo de Ecología y Conservación de Islas A.C.}{Field technician}{2012}{ }{\begin{cvitems}
\item Monitoring Eradication of Invasive Mammals: Farallón de San Ignacio and San Pedro Martir Islands
\end{cvitems}}
    \cventry{Fondo Mexicano de Conservación de la Naturaleza}{Field technician}{2010}{ }{\begin{cvitems}
\item Project: Ecological and Ecotoxicological monitoring of the Seabird community of Santa Maria Bay: establishing base lines for the use of bioindicators in decisions of environmental management, Sinaloa
\end{cvitems}}
    \cventry{Resources from basic science research SEP-CONACYT and PAPIIT}{Technical assistant}{2009}{ }{\begin{cvitems}
\item Project: Morphologic, isotopic and genetic characterization of \textit{Calidris mauri} in the Northwest of México
\end{cvitems}}
    \cventry{Pronatura River of Raptors, Cardel, Veracruz}{Education intern}{2009}{ }{\begin{cvitems}
\item Orientation to school groups, local people, and visitors
\end{cvitems}}
    \cventry{Ducks Unlimited de México (Binational Bird Conservation)}{Technical assistant}{2009}{ }{\begin{cvitems}
\item Project: Evaluation of the waterbird community in rehabilitated mangroves sites
\end{cvitems}}
    \cventry{University of Guadalajara}{Collaborator}{2008}{ }{\begin{cvitems}
\item Course: Introduction to the study of birds
\end{cvitems}}
\end{cventries}

\hypertarget{peer-reviewed-publications}{%
\section{Peer-reviewed publications}\label{peer-reviewed-publications}}

\begin{cventries}
    \cventry{\textbf{Lerma M}, Dehnhard N, Castillo-Guerrero A, Fernandez G}{Nutritional state variations in a tropical seabird throughout its breeding season}{2022}{J Comp Phy B}{}\vspace{-4.0mm}
    \cventry{\textbf{Lerma M}, Dehnhard N, Luna-Jorquera G, Voigt CC, Garthe S}{Breeding stage, not sex, affects foraging characteristics in masked boobies at Rapa Nui}{2020}{Beh Ecol Soc 74: 149}{}\vspace{-4.0mm}
    \cventry{Quispe R, \textbf{Lerma M}, Luna N, Portflitt-Toro M, Serratosa J, Luna-Jorquera G }{Foraging ranges of Humboldt penguins \textit{Spheniscus humboldti}  from Tilgo Island: the critical need for protecting a unique marine habitat. }{2020}{Mar Ornithol 48: 205}{}\vspace{-4.0mm}
    \cventry{\textbf{Lerma M}, Castillo-Guerrero JA, Hernandez-Vazquez S, Garthe S}{Foraging ecology of a marine top predator in the Eastern Tropical Pacific over 3 years with different ENSO phases}{2020}{Mar Biol 167: 88}{}\vspace{-4.0mm}
    \cventry{\textbf{Lerma M}, Castillo-Guerrero J.A., Garcia-Hernandez J, Fernandez G }{Zinc concentrations in Blue-footed booby \textit{(Sula nebouxii)} eggs, nestlings, and adults}{2020}{J Sea Research 165: 101952}{}\vspace{-4.0mm}
    \cventry{\textbf{Lerma, M}, Serratosa J, Luna-Jorquera G, Garthe S}{Foraging ecology of masked boobies \textit{(Sula dactylatra)} in the world’s largest ‘oceanic desert’}{2020}{Mar Biol 167: 87}{}\vspace{-4.0mm}
    \cventry{\textbf{Lerma M}, Castillo-Guerrero JA, Palacios E}{Non-Breeding Distribution, Abundance, and Roosting Habitat Use of the American Oystercatcher \textit{(Haematopus palliatus frazari)} in Sinaloa, Mexico}{2017}{Waterbirds 40: 95-103}{}\vspace{-4.0mm}
    \cventry{\textbf{Lerma M}, Castillo-Guerrero JA, Ruelas-Inzunza J, Fernandez, G}{Lead, cadmium and mercury in the blood of the blue-footed booby \textit{(Sula nebouxii)} from the coast of Sinaloa, Gulf of California, Mexico}{2017}{Mar Poll Bull 110: 293-298}{}\vspace{-4.0mm}
    \cventry{Castillo-Guerrero JA, \textbf{Lerma M}, Mellink E, Suazo-Guillén E, Peñaloza-Padilla EA}{Environmentally-mediated flexible foraging strategies in Brown Boobies in the Gulf of California}{2016}{Ardea 104: 33-57}{}\vspace{-4.0mm}
\end{cventries}

\hypertarget{conferences}{%
\section{Conferences}\label{conferences}}

\begin{cvhonors}
    \cvhonor{}{7th Annual Bio-Logging Science Symposium}{Virtual-Hawaii}{2021}
    \cvhonor{}{Pacific Seabird Group 45th Meeting}{La Paz, Mexico}{2018}
    \cvhonor{}{VI Reunión Internacional de Investigadores del Archipiélago Revillagigedo}{Colima, México}{2017}
    \cvhonor{}{41th Waterbirds 2017}{Reykjavik, Iceland}{2017}
    \cvhonor{}{Waterbirds 2015: Challenges and Responses}{Bar Harbor, Maine}{2015}
    \cvhonor{}{Pacific Seabird Group 40th Annual Meeting.}{Portland, US}{2013}
    \cvhonor{}{Congress for the study and conservation of the birds in Mexico}{Mazatlán, Mexico}{2011}
\end{cvhonors}

\hypertarget{tranings}{%
\section{Tranings}\label{tranings}}

\begin{cventries}
    \cventry{R-Ladies Freiburg}{From Zero to (s)Hero: Five-part introduction to R}{}{2021/Jan-May}{}\vspace{-4.0mm}
    \cventry{Falling Walls. Kiel, Germany}{Young Entrepreneurs in Science}{}{2021/Jun/1-3}{}\vspace{-4.0mm}
    \cventry{Universidad Científica del Sur. Online, Peru}{Programación en R para estudios de biologging de vertebrados marinos y sus interacciones con la actividad pesquera en el Peru}{}{2020/Nov-2021/Feb}{}\vspace{-4.0mm}
    \cventry{Max Planck Institute. Radolfzell, Germany}{AniMove }{}{2018/Sept/10-21}{}\vspace{-4.0mm}
    \cventry{Integrated School of Ocean Sciences. Kiel, Germany}{Basics in University Teaching}{}{2018/Jul/16-17}{}\vspace{-4.0mm}
    \cventry{Centre for Stable Isotope Ecology. Berlin-Brandenburg, Germany}{International Summer School on Stable Isotopes in Animal Ecology}{}{2016/Sep/12-16}{}\vspace{-4.0mm}
    \cventry{Integrated School of Ocean Sciences. Kiel, Germany}{GIS for Marine Sciences}{}{2016/Jun/20}{}\vspace{-4.0mm}
    \cventry{Integrated School of Ocean Sciences. Kiel, Germany}{Analysis of univariate data sets - first steps in R}{}{2016/Feb/8}{}\vspace{-4.0mm}
\end{cventries}

\hypertarget{relevant-skills}{%
\section{Relevant skills}\label{relevant-skills}}

\begin{tabular}[t]{>{\centering\arraybackslash}p{4cm}>{\centering\arraybackslash}p{4cm}>{\centering\arraybackslash}p{4cm}>{\centering\arraybackslash}p{4cm}}
\toprule
\textcolor[HTML]{009acd}{\textbf{Computer}} & \textcolor[HTML]{009acd}{\textbf{Laboratory}} & \textcolor[HTML]{009acd}{\textbf{Fieldwork}} & \textcolor[HTML]{009acd}{\textbf{Languages}}\\
\midrule
\textcolor[HTML]{7f7f7f}{R -- SQL -- Statistica -- SPSS -- Sigmaplot -- QGIS -- Git -- Markdown} & \textcolor[HTML]{7f7f7f}{Heavy metal analyses -- Stable isotopes -- Plasma metabolites analyses} & \textcolor[HTML]{7f7f7f}{Catching and handling wild animals -- Sample collection: extraction blood \& feathers -- Censing populations -- Monitoring breeding success} & \textcolor[HTML]{7f7f7f}{Spanish (Mother tongue) -- English (TOEFL iBT 99) -- German (Basic-A2)}\\
\bottomrule
\end{tabular}



\end{document}
